
\documentclass[12pt]{article}
\usepackage[a4paper,margin=1in]{geometry}
\usepackage{amsmath,amssymb}
\usepackage{setspace}
\usepackage{graphicx}
\usepackage{enumitem}

\onehalfspacing

\begin{document}

\begin{center}
\textbf{Lecture Scribe: CSE400 – Fundamentals of Probability in Computing}\\
\textbf{Lecture 9: Uniform, Exponential, Laplace, and Gamma Random Variables}\\
(Prepared strictly from lecture slides content)
\end{center}

\vspace{0.5cm}

\noindent\textbf{Name:} Divyaraj Vaghela\\
\textbf{Enrollment Number: AU2440251}

\section{Introduction}

This lecture discusses types of continuous random variables, focusing on:
\begin{itemize}
    \item Uniform Random Variable
    \item Exponential Random Variable
    \item Laplace Random Variable
    \item Gamma Random Variable
\end{itemize}

The lecture includes definitions, probability density functions (PDFs), cumulative distribution functions (CDFs), graphical interpretations, applications, and worked examples.

\section{Uniform Random Variable}

\subsection{Definition}

A continuous random variable $X$ is said to be uniformly distributed over the interval $[a,b]$ if its probability density function (PDF) is given by:

\[
f_X(x)=
\begin{cases}
\dfrac{1}{b-a}, & a \le x < b \\
0, & \text{elsewhere}
\end{cases}
\]

\subsection{Cumulative Distribution Function (CDF)}

The cumulative distribution function corresponding to the uniform distribution is:

\[
F_X(x)=
\begin{cases}
0, & x < a \\
\dfrac{x-a}{b-a}, & a \le x < b \\
1, & x \ge b
\end{cases}
\]

\subsection{Graphical Representation}

The PDF of a uniform random variable is constant over the interval $[a,b]$. The CDF increases linearly from $0$ to $1$ over the interval.

\section{Example 1: Uniform Random Variable}

\subsection{Problem Statement}

The phase of a sinusoid $\Theta$ is uniformly distributed over the interval $[0,2\pi]$. The PDF is given by:

\[
f_\Theta(\theta)=
\begin{cases}
\dfrac{1}{2\pi}, & 0 \le \theta < 2\pi \\
0, & \text{otherwise}
\end{cases}
\]

\subsection{Part (a): $\Pr(\Theta > 3\pi/4)$}

For a uniform random variable over $[0,2\pi]$:

\[
\Pr(a < \Theta < b)=\frac{b-a}{2\pi}
\]

Thus,

\[
\Pr(\Theta > 3\pi/4)=\frac{2\pi-3\pi/4}{2\pi}=\frac{5}{8}
\]

\subsection{Part (b): $\Pr(\Theta < \pi\mid \Theta > 3\pi/4)$}

Using conditional probability:

\[
\Pr(A\mid B)=\frac{\Pr(A \cap B)}{\Pr(B)}
\]

Let $A=\{\Theta<\pi\}$ and $B=\{\Theta>3\pi/4\}$.

\[
\Pr(3\pi/4<\Theta<\pi)=\frac{\pi-3\pi/4}{2\pi}=\frac{1}{8}
\]

Since $\Pr(B)=\frac{5}{8}$,

\[
\Pr(\Theta<\pi\mid\Theta>3\pi/4)=\frac{1/8}{5/8}=\frac{1}{5}
\]

\subsection{Part (c): $\Pr(\cos\Theta<1/2)$}

The values of $\Theta$ satisfying $\cos\Theta=1/2$ are:

\[
\Theta=\frac{\pi}{3},\; \frac{5\pi}{3}
\]

Thus,

\[
\cos\Theta<\frac{1}{2} \quad \text{for} \quad \frac{\pi}{3}<\Theta<\frac{5\pi}{3}
\]

Therefore,

\[
\Pr(\cos\Theta<1/2)=\frac{5\pi/3-\pi/3}{2\pi}=\frac{2}{3}
\]

\section{Applications of Uniform Random Variable}

\begin{itemize}
    \item Phase of a sinusoidal signal when angles between $0$ and $2\pi$ are equally likely
    \item Random number generation between $0$ and $1$ for simulations
    \item Arrival time of a user within a known time window assuming no preference
\end{itemize}

\section{Exponential Random Variable}

\subsection{Definition}

The exponential random variable has PDF and CDF defined for any $b>0$.

\subsection{Probability Density Function}

\[
f_X(x)=\frac{1}{b}e^{-x/b}u(x)
\]

where $u(x)$ denotes the unit step function.

\subsection{Cumulative Distribution Function}

\[
F_X(x)=[1-e^{-x/b}]u(x)
\]

\subsection{Graphical Interpretation}

The PDF decreases exponentially as $x$ increases. The CDF increases monotonically and approaches $1$ asymptotically.

\section{Example 2: Exponential Random Variable}

\subsection{Problem Statement}

Let $X$ be an exponential random variable with PDF:

\[
f_X(x)=e^{-x}u(x)
\]

\subsection{Part (a): $\Pr(3X<5)$}

Since $3X<5 \Rightarrow X<5/3$,

\[
\Pr(3X<5)=\Pr(X<5/3)=1-e^{-5/3}
\]

\subsection{Part (b): Generalization}

For arbitrary constant $y$:

\[
\Pr(3X<y)=\Pr(X<y/3)=1-e^{-y/3}
\]

\section{Remaining Topics in Lecture Outline}

The lecture outline lists the following additional topics:
\begin{itemize}
    \item Laplace Random Variable
    \item Gamma Random Variable
    \item Graph and Special Cases
    \item Example
    \item Homework Problem
    \item Problem Solving and In-class Activity
\end{itemize}

Detailed derivations of these topics are not provided within the given slides.

\section{Logical Progression of Concepts}

\begin{enumerate}
    \item Introduction to uniform distribution and its properties
    \item Application-based problem solving using uniform distribution
    \item Introduction to exponential distribution
    \item Application-based problems using exponential distribution
    \item Preview of additional continuous distributions
\end{enumerate}

\section{Summary of Key Mathematical Results}

\subsection*{Uniform Random Variable}

\[
f_X(x)=
\begin{cases}
\dfrac{1}{b-a}, & a \le x < b \\
0, & \text{otherwise}
\end{cases}
\]

\[
F_X(x)=
\begin{cases}
0, & x < a \\
\dfrac{x-a}{b-a}, & a \le x < b \\
1, & x \ge b
\end{cases}
\]

\subsection*{Exponential Random Variable}

\[
f_X(x)=\frac{1}{b}e^{-x/b}u(x)
\]

\[
F_X(x)=[1-e^{-x/b}]u(x)
\]

\end{document}

