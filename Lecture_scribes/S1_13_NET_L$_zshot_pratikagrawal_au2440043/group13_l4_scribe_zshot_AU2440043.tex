\documentclass[12pt]{article}
\usepackage{amsmath, amssymb}
\usepackage{geometry}
\usepackage{setspace}
\geometry{a4paper, margin=1in}
\setstretch{1.2}

\begin{document}

\section*{Lecture Scribe — CSE400}

\textbf{Lecture 4: Joint Probability and Conditional Probability}\\
\textbf{Instructor:} Dhaval Patel, PhD\\
\textbf{Date:} January 15, 2026

\hrule
\vspace{1em}

\section*{1. Introduction to Probability Theory}

\subsection*{1.1 Experiments, Sample Space, and Events}

\textbf{Definitions}

\begin{itemize}
    \item \textbf{Experiment (E):} A procedure we perform that produces some result.\\
    \textit{Example:} Tossing a coin five times.

    \item \textbf{Outcome ($\omega$):} A possible result of an experiment.\\
    \textit{Example:} One outcome of five coin tosses: \texttt{HHTHT}.

    \item \textbf{Event (A, B, C, …):} A set of outcomes of an experiment.\\
    \textit{Example:} Event ( C ): all outcomes consisting of an even number of heads.

    \item \textbf{Sample Space (S):} The set of all possible distinct outcomes of an experiment.
\end{itemize}

Outcomes in ( S ) are:
\begin{itemize}
    \item \textbf{Mutually Exclusive:} Only one can occur at a time.
    \item \textbf{Collectively Exhaustive:} No other outcome is possible.
\end{itemize}

Sample space can be:
\begin{itemize}
    \item Discrete
    \item Countably infinite
    \item Continuous
\end{itemize}

\textbf{Examples}
\begin{itemize}
    \item Flipping a fair coin once
    \item Rolling a cubical die
    \item Rolling two dice
    \item Flipping a coin until tails occurs
    \item Random number generator on ([0,1))
\end{itemize}

\section*{2. Axioms of Probability}

Probability is a function that assigns a numerical measure to events.

\textbf{Axioms}

\begin{enumerate}
    \item For any event ( A ):
    \[
    0 \leq Pr(A) \leq 1
    \]

    \item If ( S ) is the sample space:
    \[
    Pr(S) = 1
    \]

    \item If ( A \cap B = \varnothing ):
    \[
    Pr(A \cup B) = Pr(A) + Pr(B)
    \]
\end{enumerate}

For infinite mutually exclusive events ( $A_i$ ):
\[
Pr\left(\bigcup_{i=1}^{\infty} A_i \right) = \sum_{i=1}^{\infty} Pr(A_i)
\]

\section*{3. Corollary from Axioms}

\textbf{Corollary 2.1 (Finite case)}\\
For a finite number of mutually exclusive events ( $A_i$ ):
\[
Pr\left(\bigcup_{i=1}^{n} A_i \right) = \sum_{i=1}^{n} Pr(A_i)
\]

\section*{4. Propositions from Axioms}

\begin{itemize}
    \item \textbf{Proposition 2.1}
    \[
    Pr(A^c) = 1 - Pr(A)
    \]

    \item \textbf{Proposition 2.2}\\
    If ( A $\subset$ B ), then
    \[
    Pr(A) \leq Pr(B)
    \]

    \item \textbf{Proposition 2.3}
    \[
    Pr(A \cup B) = Pr(A) + Pr(B) - Pr(A \cap B)
    \]

    \item \textbf{Proposition 2.4 (Inclusion–Exclusion Principle)}\\
    \[
    Pr(A_1 \cup A_2 \cup \dots \cup A_n)
    \]
    expressed using alternating sums of intersections.
\end{itemize}

\section*{5. Assigning Probabilities}

\subsection*{5.1 Classical Approach}

\begin{itemize}
    \item Coin flip: ( Pr(H) = Pr(T) )
    \item Die roll: ( Pr(\text{even}) )
    \item Two dice: ( Pr(\text{sum} = 5) )
\end{itemize}

\subsection*{5.2 Relative Frequency Approach}

\[
Pr(A) = \lim_{n \to \infty} \frac{\text{Number of times A occurs}}{n}
\]

Limitations:
\begin{itemize}
    \item Requires infinite repetitions
    \item Many phenomena are not repeatable
\end{itemize}

\section*{6. Joint Probability}

\subsection*{6.1 Motivation}

Events are not always mutually exclusive. We are interested in:
\[
Pr(A \cap B)
\]

\subsection*{6.2 Notation}

\[
Pr(A, B) \quad \text{or} \quad Pr(A \cap B)
\]

For multiple events:
\[
Pr(A_1, A_2, \dots, A_n)
\]

\subsection*{6.3 Calculation Approaches}

\begin{itemize}
    \item \textbf{Classical:} Express events using atomic outcomes and find common outcomes.
    \item \textbf{Relative Frequency:}
    \[
    Pr(A,B) = \frac{n_{A,B}}{n}
    \]
\end{itemize}

\subsection*{Example 1: Card Deck Example}

52 cards, 13 per suit.

Let:
\begin{itemize}
    \item ( A ): Red card
    \item ( B ): Number card (Ace included)
    \item ( C ): Heart card
\end{itemize}

\textbf{Individual Probabilities}
\[
Pr(A) = \frac{26}{52} = \frac{1}{2}
\]
\[
Pr(B) = \frac{40}{52} = \frac{5}{26}
\]
\[
Pr(C) = \frac{13}{52} = \frac{1}{4}
\]

\textbf{Joint Probabilities}
\[
Pr(A,B) = \frac{20}{52}
\]
\[
Pr(A,C) = \frac{13}{52}
\]
\[
Pr(B,C) = \frac{10}{52}
\]

\subsection*{Example 2: Costume Party}

Tops: 3 T-shirts, 1 cape → ( Pr(\text{Cape}) = $\frac{1}{4}$ )

Bottoms: 2 pants, 4 boxers → ( Pr(\text{Boxers}) = $\frac{4}{6}$ )

Joint probability:
\[
Pr(\text{Cape and Boxers}) = \frac{1}{4} \times \frac{4}{6} = \frac{1}{6}
\]

Correct option: C

\section*{7. Conditional Probability}

\subsection*{7.1 Motivation}

Occurrence of one event may depend on another.

\subsection*{7.2 Definition}

\[
Pr(A|B) = \frac{Pr(A,B)}{Pr(B)}, \quad Pr(B) > 0
\]

\subsection*{7.3 Product Rule}

\[
Pr(A,B) = Pr(A|B)Pr(B) = Pr(B|A)Pr(A)
\]

\subsection*{7.4 Three Events}

\[
Pr(A,B,C) = Pr(C|A,B)Pr(B|A)Pr(A)
\]

\subsection*{7.5 Chain Rule}

\[
Pr(A_1, A_2, \dots, A_n) =
Pr(A_n | A_1,\dots,A_{n-1}) \dots Pr(A_2|A_1)Pr(A_1)
\]

\subsection*{Example 3: Cards Without Replacement}

Two cards drawn, no replacement.

\begin{itemize}
    \item ( A ): First card is Spade
    \item ( B ): Second card is Spade
\end{itemize}

After one spade removed:
\begin{itemize}
    \item Remaining cards = 51
    \item Remaining spades = 12
\end{itemize}

\[
Pr(B|A) = \frac{12}{51}
\]

\subsection*{Example 4: Game of Poker — Flush}

\textbf{Flush in Spades}
\[
Pr = \frac{13}{52} \times \frac{12}{51} \times \frac{11}{50} \times \frac{10}{49} \times \frac{9}{48}
\]

\textbf{Any Flush}
\[
4 \times Pr(\text{Spade Flush})
\]

\subsection*{Example 5: The Missing Key}

Given:
\[
P(L) = 0.4,\quad P(R) = 0.4,\quad P(K)=0.8
\]

Find:
\[
P(R | L^c)
\]

Since key cannot be in both pockets:
\[
P(R|L^c) = \frac{P(R)}{P(L^c)} = \frac{0.4}{0.6} = \frac{2}{3}
\]

Correct option: C

\vspace{1em}
\hrule
\vspace{1em}

\textbf{End of Lecture}

\end{document}
