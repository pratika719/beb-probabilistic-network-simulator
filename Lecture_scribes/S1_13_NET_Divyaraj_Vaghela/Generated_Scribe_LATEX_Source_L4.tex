\documentclass[12pt]{article}
\usepackage[a4paper, margin=1in]{geometry}

\begin{document}

\noindent
\textbf{\Large CSE400 -- Fundamentals of Probability in Computing}

\vspace{0.3cm}

\noindent
\textbf{\large Lecture 4: Joint Probability and Conditional Probability}

\vspace{0.3cm}

\noindent
\textbf{Instructor:} Dhaval Patel, PhD

\noindent
\textbf{Date:} January 15, 2026

\vspace{0.3cm}

\noindent


\vspace{0.5cm}
\hrule
\vspace{0.5cm}

\section*{1. Introduction}

\subsection*{1.1 Course Motivation and Applications}

\textbf{Engineering Applications}

\vspace{0.3cm}

The lecture presents engineering applications where probabilistic reasoning is used:

\begin{enumerate}
    \item Speech Recognition System
    \begin{itemize}
        \item The system uses vocabulary templates such as:
        \begin{itemize}
            \item Hello
            \item Yes
            \item No
            \item Bye
        \end{itemize}
    \end{itemize}
\end{enumerate}

\begin{enumerate}
    \setcounter{enumi}{1}

    \item Radar System
    \begin{itemize}
        \item Radar systems detect objects such as aircraft using transmitted and received signals.
    \end{itemize}

    \item Communication Network
    \begin{itemize}
        \item Communication networks involve transmission of data from source nodes to destination nodes through interconnected systems.
    \end{itemize}
\end{enumerate}

\vspace{0.5cm}
\hrule
\vspace{0.5cm}

\section*{2. Lecture Outline}

The lecture includes the following major components:

\subsection*{2.1 Introduction to Probability Theory}

\begin{itemize}
    \item Experiments, Sample Space, and Events
    \item Axioms of Probability
    \item Corollaries and Propositions from Probability Axioms
    \item How to Assign Probability: Classical and Relative Frequency Approaches
\end{itemize}

\subsection*{2.2 Joint Probability}

\begin{itemize}
    \item Motivation, Notation, and Concepts of Joint Probability
    \item Example 1: Card Deck Example
    \item Example 2: Costume Party Example
\end{itemize}
\subsection*{2.3 Conditional Probability}

\begin{itemize}
    \item Motivation, Notation, and Concepts of Conditional Probability
    \item Example 3: Cards Without Replacement
    \item Example 4: Game of Poker
    \item Example 5: The Missing Key
\end{itemize}

\vspace{0.5cm}
\hrule
\vspace{0.5cm}

\section*{3. Introduction to Probability Theory}

\subsection*{3.1 Experiments, Sample Space, and Events}

\subsubsection*{3.1.1 Experiment}

\textbf{Definition:}

An \textbf{Experiment (E)} is a procedure we perform that produces some result.

\vspace{0.3cm}

\textbf{Example:}

\begin{itemize}
    \item Tossing a coin five times
    \item Denoted as $E_5$
\end{itemize}
\subsubsection*{3.1.2 Outcome}

\textbf{Definition:}

An \textbf{Outcome ($\xi$)} is a possible result of an experiment.

\vspace{0.3cm}

\textbf{Example:}

For experiment $E_5$:

\begin{itemize}
    \item One possible outcome is
\end{itemize}

\[
\xi_1 = HHTHT
\]

\vspace{0.5cm}
\hrule
\vspace{0.5cm}

\subsubsection*{3.1.3 Event}

\textbf{Definition:}

An \textbf{Event (any letter)} is a certain set of outcomes of an experiment.

\vspace{0.3cm}

\textbf{Example:}

Consider event $C$ with experiment $E_5$:

\[
C = \{\text{all outcomes consisting of an even number of heads}\}
\]
\subsection*{3.2 Sample Space}

\subsubsection*{3.2.1 Definition}

The \textbf{Sample Space (S)} is the collection or set of all possible distinct outcomes of an experiment.

\vspace{0.3cm}

The outcomes in the sample space must satisfy:

\begin{enumerate}
    \item \textbf{Mutually Exclusive}
    \begin{itemize}
        \item Two outcomes cannot occur simultaneously.
        \item Example: In a coin flip, you can get heads or tails but not both.
    \end{itemize}

    \item \textbf{Collectively Exhaustive}
    \begin{itemize}
        \item All possible outcomes must be included.
        \item Example: In a coin flip, outcomes are only heads or tails.
    \end{itemize}
\end{enumerate}

\vspace{0.5cm}
\hrule
\vspace{0.5cm}

\subsubsection*{3.2.2 Properties of Sample Space}

\begin{itemize}
    \item $S$ is the universal set of outcomes of an experiment.
    \item Sample space can be:
    \begin{itemize}
        \item Discrete
        \item Countably infinite
        \item Continuous
    \end{itemize}
\end{itemize}
\subsubsection*{3.2.3 Examples of Sample Spaces}

\begin{enumerate}
    \item Flipping a fair coin once
    \item Rolling a cubical die with numbered faces
    \item Rolling two dice
    \item Flipping a coin until a tail occurs
    \item Random number generator with interval $[0,1]$
\end{enumerate}

\vspace{0.5cm}
\hrule
\vspace{0.5cm}

\section*{4. Axioms of Probability}

\subsection*{4.1 Probability Definition}

Probability is defined as:

\begin{itemize}
    \item A measure of the likelihood of various events
    
    OR
    
    \item A function of an event that produces a numerical quantity measuring the likelihood of that event.
\end{itemize}

There are many ways to define such a function.
\subsection*{4.2 Probability Axioms}

The axioms are statements that are taken to be self-evident and require no proof.

\vspace{0.3cm}

\textbf{Axiom 1:}

For any event $A$,

\[
0 \leq Pr(A) \leq 1
\]

\vspace{0.5cm}
\hrule
\vspace{0.5cm}

\textbf{Axiom 2:}

If $S$ is the sample space for a given experiment,

\[
Pr(S) = 1
\]

\vspace{0.5cm}
\hrule
\vspace{0.5cm}

\textbf{Axiom 3 (Finite Additivity):}

If $A \cap B = \emptyset$,

\[
Pr(A \cup B) = Pr(A) + Pr(B)
\]
\vspace{0.5cm}
\hrule
\vspace{0.5cm}

\textbf{Axiom 4 (Countable Additivity):}

For an infinite number of mutually exclusive sets $A_i$,

where

\[
A_i \cap A_j = \emptyset \quad \text{for all } i \neq j
\]

Then:

\[
Pr \left( \bigcup_{i=1}^{\infty} A_i \right) = \sum_{i=1}^{\infty} Pr(A_i)
\]

\vspace{0.5cm}
\hrule
\vspace{0.5cm}

\section*{5. Corollaries from Probability Axioms}

\textbf{Definition}

Let $A$ be the null event for all values of $i$ greater than $n$.
\subsection*{Corollary 2.1}

For $M$ finite number of mutually exclusive sets $A_i$,

\[
A_i \cap A_j = \emptyset \quad \text{for all } i \neq j
\]

Then:

\[
Pr \left( \bigcup_{i=1}^{M} A_i \right) = \sum_{i=1}^{M} Pr(A_i)
\]

\vspace{0.5cm}
\hrule
\vspace{0.5cm}

\textbf{Notes}

\begin{itemize}
    \item Axiom 3 is equivalent to Corollary 2.1 when the sample space is finite.
    \item The generality of Axiom 3 is necessary when the sample space contains an infinite number of points.
\end{itemize}
\section*{6. Propositions from Probability Axioms}

\subsection*{Proposition 2.1}

For any event $A$,

\[
Pr(A^c) = 1 - Pr(A)
\]

\vspace{0.5cm}
\hrule
\vspace{0.5cm}

\subsection*{Proposition 2.2}

If $A \subset B$, then:

\[
Pr(A) \leq Pr(B)
\]
\section*{7. Logical Flow of Concepts}

The lecture establishes the following conceptual progression:

\begin{enumerate}
    \item Define experiments and outcomes
    \item Construct events as sets of outcomes
    \item Define the sample space as the universal outcome set
    \item Introduce probability as a measure over events
    \item Establish probability axioms
    \item Derive corollaries and propositions from axioms
    \item Use these foundations to study joint probability and conditional probability (introduced later in lecture outline)
\end{enumerate}


\end{document}
